%% full citations in body of text without reference section:
%%https://tex.stackexchange.com/questions/49048/how-to-cite-one-bibentry-in-full-length-in-the-body-text
%% print DOIs and URLs in full citations:
%%https://tex.stackexchange.com/questions/3802/how-to-get-doi-links-in-bibliography

\documentclass[12pt]{article}
%%\usepackage{filecontents}
\usepackage[margin=1in]{geometry}
\usepackage[colorlinks=TRUE, urlcolor=blue]{hyperref}
\usepackage{xspace}

%% macros
\newcommand{\R}{{\mathcal R}}
\newcommand{\etal}{\emph{et al.}\/\xspace}

%% bibliography
\usepackage{natbib}
\usepackage{bibentry}
\nobibliography*

\parindent=0pt
\parskip=12pt

\title{Possible Sources for Presentations}
\author{Mathematics 747, Fall 2020}

\begin{document}

\maketitle

\emph{This is by no means an exhaustive list.  You are welcome to
  suggest other papers etc.}

\section*{Classic papers}

\subsubsection*{Daniel Bernoulli's analysis of smallpox mortality and variolation (1766)}

\emph{Original paper:}\\
\bibentry{Bern1766}

\emph{English translation and commentary:}\\
\bibentry{BlowBern04}

\emph{Updated analysis:}\\
\bibentry{DietHees02}

\subsubsection*{The SIR model: Kermack and McKendrick (1927)}

\emph{Original paper:}\\
\bibentry{KermMcKe27}

\emph{Commentary:}\\
\bibentry{Diek+95}

\subsubsection*{Nonlinear incidence}

\emph{Original proposal of $\beta S^hI$ models:}\\
\bibentry{WilsWorc45}

\emph{Analysis via bifurcation theory:}\\
\bibentry{Liu+87}

\section*{Reviews}

\subsubsection*{Vaccines}

\href{https://www.historyofvaccines.org/timeline/all}{``History of
  Vaccines'' web site} by the College of Physicians of Philadelphia.

\subsubsection*{Infectious disease dynamics}

\emph{Major review:}\\
\bibentry{Hees+15}

\emph{Commentary during COVID-19 pandemic:}\\
\bibentry{Metc+20}

\section*{Online lectures}

\subsubsection*{Science communication in relation to epidemics}

\href{https://www.youtube.com/watch?v=ZSp-k6C9qBw}{``Misinfodemic
  2020''}: \href{http://ctbergstrom.com/}{Carl Bergstrom}'s plenary
talk at the Society for Mathematical Biology (SMB) 2020 annual
meeting.

\section*{Estimating the initial epidemic growth rate $r$}

\bibentry{Ma+14}

\bibentry{Earn+20}

\section*{Estimating the basic reproduction number $\R_0$}

\emph{Finding formulae for $\R_0$ from ODE models:}\\
\bibentry{vandWatm02}

\emph{From $r$ to $\R_0$:}\\
\bibentry{WallLips07}

\emph{Understanding generation intervals, which are needed to go from
  $r$ to $\R_0$:}\\
\bibentry{ChamDush15}

\emph{On COVID-19 $\R_0$ estimation:}\\
\bibentry{Park+20}

\emph{Review of COVID-19 $\R$ and $r$ estimation for the UK:}\\
\bibentry{RoySoc20}

\section*{Estimating the effective reproduction number $\R_t$}

\bibentry{WallTeun04}

\bibentry{Gold+09}

\bibentry{Cori+13}

\subsubsection*{COVID-19 $\R_t$ estimations}

\bibentry{Pan+20}

\href{https://github.com/keyajoshi/Pan_response}{Lipsitch \etal
  comment on Pan \etal JAMA paper}
{\footnotesize\parskip=0pt
\begin{itemize}\itemsep0pt
\item notes that Pan \etal would not share their data, so they digitized the graph
\item Cori \etal (\texttt{EpiEstim}) does not use WT method
\end{itemize}
}

\href{https://www.datacamp.com/community/tutorials/replicating-in-r-covid19}{Data camp on estimating $\R_t$ for COVID-19}.

\href{https://staff.math.su.se/hoehle/blog/2020/04/15/effectiveR0.html}{Michael Höhle.  Effective reproduction number estimation}.

\section*{COVID-19 papers and resources}

\bibentry{Li+20}

\bibentry{Fauc+20}

\href{https://github.com/mac-theobio/Lab_meeting/blob/master/covid-19/README.md}{DE's COVID links page}

%%%%%%%%%%%%%%%%%%
%% BIBLIOGRAPHY %%
%%%%%%%%%%%%%%%%%%

\bibliographystyle{plainnat}
%%\bibliography{\jobname.bib}
\nobibliography{\jobname.bib} % do not print a reference section

\end{document}
